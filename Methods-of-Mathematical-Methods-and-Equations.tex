\documentclass[a4paper,12pt,parskip=full,BCOR=1cm]{scrreprt}
\usepackage{graphicx}
\usepackage[arrowdel]{physics}
\usepackage[hidelinks]{hyperref}
\usepackage{nicefrac}
\usepackage{amsmath}
\usepackage{amssymb}
\usepackage{amsfonts}
\usepackage{relsize}
\usepackage{booktabs}
\KOMAoption{chapterprefix}{true}

\renewcommand*\raggedchapter{\centering}
\RedeclareSectionCommand[beforeskip=0pt,afterskip=2\baselineskip]{chapter}
\setkomafont{chapterprefix}{\normalsize\mdseries}

\renewcommand*{\chapterformat}{%
  \chapappifchapterprefix{\nobreakspace}\thechapter\autodot%
  \IfUsePrefixLine{%
    \par\nobreak\vspace{-\parskip}\vspace{-.6\baselineskip}%
    \rule{0.9\textwidth}{.2pt}%
  }{}%
}

\usepackage{mathtools}
\DeclarePairedDelimiter\ceil{\lceil}{\rceil}
\DeclarePairedDelimiter\floor{\lfloor}{\rfloor}
\DeclarePairedDelimiter\integerpart{\[}{\]}
\usepackage{mathrsfs}
\newcommand{\fourier}[1]{\ensuremath{\mathlarger{\mathcal{F}}\!\!\left[#1 \right]}}
\newcommand{\inversefourier}[1]{\ensuremath{\mathlarger{\mathcal{F}^{-1}}\!\!\left[#1 \right]}}
\newcommand{\laplace}[1]{\ensuremath{\mathlarger{L}\!\!\left[#1 \right]}}
\DeclareMathOperator{\erfc}{erfc}

\setkomafont{disposition}{\rmfamily\scshape\bfseries}


\usepackage[largesc]{newpxtext}
\usepackage{newpxmath}
\setkomafont{descriptionlabel}{\rmfamily\bfseries}
\linespread{1.05}
\setcounter{tocdepth}{1}
\usepackage[tikz]{mdframed}
\newmdenv[frametitle=Example,roundcorner=5pt]{example}
\newmdenv[frametitle=Note,roundcorner=5pt]{note}

\title{Methods of Mathematical Physics}
\subtitle{Methods and Madness}
\author{James Shaw}
\date{Winter 2017}
\renewcommand{\leq}{\leqslant}
\renewcommand{\geq}{\geqslant}
\renewcommand{\vec}{\underline}
\setcounter{tocdepth}{0}

\begin{document}
\begin{titlepage}
\begin{minipage}[b]{0.6\textwidth}
\usekomafont{disposition}
        {\huge School of Physics\\ and Astronomy\\}
\end{minipage}
\hfill
\begin{minipage}[t]{40mm}
        \includegraphics[width=40mm]{crest.eps}
\end{minipage}
\vspace*{4cm}
\begin{center}
{\large\usekomafont{title} Methods of Mathematical Physics\\}
{\Huge\usekomafont{title}Maths, Methods and Madness\\}\vspace{4cm}
{\Large {\bfseries Lecturer:} Dr.~Kristel Torokoff\\}\vspace{1cm}
{\LARGE\bfseries 2017-18\\}
\end{center}
\vspace{\fill}

\end{titlepage}

\tableofcontents
\part{The Standard Book of Maths, Grade IV}
\chapter{Infinite Series And Asymptotic Expansions}
\section{Infinite Series}
An infinite series $\sum_{n=0}^\infty a_n$ converges to some $S$ if $$\forall \epsilon > 0, \exists N_0 \quad\textrm{such that}\quad \abs{S - \sum_{n=0}^N a_n} < \epsilon \quad\forall N< N_0$$ and is absolutely convergent if $\sum_{n=0}^\infty \abs{a_n}$ is convergent.

\begin{example}
 The geometric series is absolutely convergent to $$\sum_{n=0}^\infty x^n = \frac{1}{1-x}$$
\end{example}
If we want to prove (absolute) convergence, we can use
\begin{equation*}
 \textrm{If}\quad\lim_{n\rightarrow \infty}\abs{\frac{a_{n+1}}{a_n}} < 1\quad \textrm{then the series}\quad\sum_{n=0}^\infty a_n \quad \textrm{converges absolutely}.
\end{equation*}

We can also use the root test ($\lim_{n\rightarrow\infty} (\abs{a_n})^{\nicefrac{1}{n}}$) or comparison with a sensible integral where if $f(x)$ is monotonically decreasing, it satisfies
$$ \sum_{n=b}^\infty f(n) \leq \int_{b-1}^\infty f(x)\dd{x} $$ if it is convergent.
An equivalent definition for the integral test works for testing divergence.

\begin{example}
 Some useful ready-made series are
 \begin{align*}
  \frac{1-x^N}{1-x} & = \sum_{n=0}^N x^n                                    & \frac{1}{1-x} & = \sum_{n=0}^\infty x^n,\quad \abs{x} < 1                         \\
  \exp(x)           & = \sum_{n=0}^{\infty} \frac{x^n}{n!}                  & \log(1+x)     & = \sum_{n=0}^{\infty}\frac{(-1)^{n+1}}{n}x^n, \quad -1 < x \leq 1 \\
  \sin(x)           & = \sum_{n=0}^{\infty}\frac{(-1)^{n}}{(2n+1)!}x^{2n+1} & \cos(x)       & = \sum_{n=0}^{\infty}\frac{(-1)^{n}}{(2n)!}x^{2n}
 \end{align*}
\end{example}

\section{Asymptotic Series}
We define asymptotic expansion as
$$ S_n= \sum_{s=0}^n a_s z^{-s}$$ for $z \rightarrow \infty$ if the remainder satisfies
$$ \lim_{z\rightarrow\infty} z^n \qty[f(z) - \sum_{s=0}^n a_s z^{-s}] = 0.$$
This is written formally as $f(z) \sim S_n(z)$ (often abused horrifically).

A lot of the time, asymptotic expansions appear due to swapping integrals and sums in places where this is not 100\% allowed --- if the sum is divergent but the integral convergent, for example.
The basic point here is to find some value of $n$ which minimises the remainder for a value of $z$.
A given series can be the asymptotic series for several different functions, but a function has a unique asymptotic series.

\chapter{Complex Analysis}
\section{Basics}
Complex differentiability of a function in the form $f(z) = u(z) + iv(z)$ where $z=x+iy$ is defined by the Cauchy-Riemann conditions $$ \pdv{u}{x}= \pdv{v}{y} \quad \textrm{and} \quad \pdv{u}{y} = - \pdv{v}{x} $$ and analyticity on a domain is defined as $f(x)$ is analytic if it is differentiable on the domain.

If a function is analytic in a region entirely enclosed by a contour, we find that
\begin{equation*}
 \oint_C f(z) \dd{z} = 0
\end{equation*}
which allows us to deform a contour in any way we wish so long as it doesn't move out of the analytic domain to do so.
This also leads us to the Cauchy Integral Formula
\begin{equation*}
 f(z) = \frac{1}{2\pi i}\oint_C \frac{f(\zeta)}{\zeta-z} \dd{\zeta}.
\end{equation*}
Differentiating this $n$ times with respect to $z$ gives the link between complex differentiation and integration $$ \dv[n]{f}{z} = \frac{n!}{2\pi i} \oint \frac{f(\zeta)}{(\zeta - z)^{n+1}} \dd{\zeta}$$ thus if a function is differentiable in a region, it is infinitely differentiable.

\section{Series Expansions}
Any complex-valued function, in its analytic region, can be written as a pure Taylor series.

By adding terms in $(z-z_0)^{-n}$, we add areas of non-analyticity to the function (`singularities') of which there are three types:
\begin{description}
 \item [Removable Singularity] A singularity which can be defined out piecewise sensibly such that the function is then analytic
 \item [Isolated Singularity (Pole)] A singularity of finite order which is unremovable.
Simple poles have 1st order Laurent expansions (only a $\frac{1}{z-z_0}$ term).
 \item [Essential singularity] A singularity of infinite order in its Laurent expansion (e.g.~$e^{\frac{1}{x}}=\sum_0^\infty \frac{1}{n!x^n}$)
\end{description}

The coefficient $a_{-1}$ which blows up at $z=z_0$ is called the residue at $z_0$.
This leads us to the residue theorem, which states
\begin{equation*}
 \oint_{\partial C} f(z) \dd{z} = 2\pi i \sum_{i}\Res(f,a_i)
\end{equation*}
for all $a_i\in f(C)$ for $\partial C$ a closed simply connected anticlockwise contour.

Finding residues is a bit of a nightmare.
Fundamentally, you're trying to get the 1st term in the Laurent expansion in a partial fractions form.

When doing an integral if you can close it in the complex plane, it's often easier to calculate thanks to the residue theorem.
We can then often remove the extra integrals with bounding.

A useful lemma for this is Jordan's Lemma.
If $I_n$ is an integral of the form $$I_n = \int _C f(z)e^{ikz} \dd{z}$$ for $k\in \mathbb{R^+}$ an $C$ a semicircular contour in the upper plane, then provided $\lim_{\abs{z} \rightarrow \infty}f(z) = 0$ and $0\leq\arg(z)\leq\pi$, then Jordan's Lemma states that $$\lim_{R\rightarrow \infty}I_R = 0.$$

\section{Multivalued Functions (and clever contour choices)}
Anywhere where there's an $\arg$, there's a choice of where you break for the periodic repeat in order to make your multivalued map injective.
The `principle' branch is the choice of $\theta = \pm \pi$.
Introducing a branch cut separates your complex plane into Riemann sheets.

Complex powers cause branch cuts.
A complex power is defined as
\begin{align*}
 z^p & = \exp(p\log(z))                          \\
     & = \exp(p\log{Re^{i\theta}})               \\
     & = \exp(p\log{R} + p\log{e^{i\theta}})     \\
     & = R^p\{pi\theta + pn\pi|n\in\mathbb{N}\}.
\end{align*}
This gives us a set, which can either be finite, countably infinite or uncountably infinite (depending on the $p$).

The introduction of branch cuts also means that contours we choose start to become more complicated.
Such contours as the `keyhole' and `dogbone' contours are used in situations where we need to navigate around the branch cut.

The choice of contour when extending an integral into the complex plane is entirely arbitrary.
We will usually choose it to ensure convergence of the integral (if the integral is of the form $\int_C e^{ikx}f(x))\dd{x}$ we choose a contour in the upper half-plane, such that the integral is always negative, and we can use Jordan's lemma).

\section{Analytic Continuation}
The Identity Theorem states that if two functions $f_1(z)$ and $f_2(z)$ are analytic in a region $R$, then if $f_1(W)=f_2(W)$ where $W$ is a simply connected subregion or curve in $R$, then $f_1(z)=f_2(z)$ everywhere in $R$.

Using this theorem, we can take an expansion, find a closed form for it, expand it about a different point and repeat.
Often, we'll be able to show that a function's closed form is analytic for all of the plane except for its poles.

\chapter{The Gamma Function}
\section{The Factorial to the Gamma}
We define the factorial as usual
$$n! = n(n-1)\cdots1 \quad \textrm{with}\quad 0! = 1.$$
We can derive (using integration by parts) the integral representation of the factorial $$I(n) = \int_0^\infty e^{-t}t^n\dd{t}\quad n\in\mathbb{N}$$ which through analytic continuation we extend $n$ to a complex number $z\in\mathbb{C}$ where $\Re(Z)>-1$ which gives us $$\Gamma (z+1) = \int_0^\infty e^{-t}t^z\dd{t}.$$

\section{More Clever Contours}
In order to extend $\Gamma(z)$ even further into the complex plane, we use the Hankel contour, which is a three part contour around the positive branch cut, \begin{align*}
 C   & = C_1 + C_2 + C_3                                                                                                                                                        \\
 C_1 & : \lim_{\varepsilon\rightarrow0}\int_{\infty + i\varepsilon}^{0+ i\varepsilon}\dd{t}                                                                                     \\
 C_2 & :\lim_{\varepsilon\rightarrow0}\int^{e^{2\pi i}\infty - i\varepsilon}_{e^{2\pi i}-i\varepsilon}\dd{t} \intertext{remembering that there's a branch cut in the way, and }
 C_3 & : \lim_{\delta=0} \int_0^{2\pi}\dd{\theta} \text{ where } t = \delta e^{i\theta}.
\end{align*}
When this is evaluated, this gives us the continuation to almost all $z$ of $$2\sin(\pi z)\Gamma(z+1)=\int_C e^{-t}(-t)^z \dd{t}.$$ The branch that arises due to the power in the exponential changes in nature depending on the $z$.
If $z$ is a positive integer, we recover the factorial again.

\section{The Beta Function}
By considering the product of two $\Gamma$ functions, we derive the $B$ function
$$ B(r,s) = \frac{\Gamma(r)\Gamma(s)}{\Gamma(r+s)}= \int_0^1 t^{r-1}(1-t)^{s-1}\dd{t}$$ In the interests of convergence (this is a common theme) we require that $\Re(r) > 0$ and $\Re(s) > 0$.

Now if we consider
\begin{align*}
 \Gamma(z)\Gamma(1-z) & = \Gamma(1)B(1,1-z)               \\
                      & = \int_0^1 t^{z-1}(1-t)^{z}\dd{t} \\
                      & =\frac{\pi}{\sin\pi z},
\end{align*}
this is Euler's reflection formula.
This gives us the Beta function analytic continuation to all $z \in \mathbb{C}-\mathbb{N}$.
This also gives us $\Gamma(\frac{1}{2})=\sqrt\pi$.

\chapter{Approximating Difficult Integrals with Solvable Gaussians}
\section{Laplace's Method}
Laplace's Method is contingent upon the concept that an integral in the form $$I(x) = \int_a^b f(t) e^{x\phi(t)}\dd{t}$$ where $f(t)$ is slowly varying and $\phi$ is real, analytic and has a maximum at $t=c$ is dominated by the region around the maximum.
We can therefore approximate it and not pick up too much error, and have it actually integrate.

This is a three (and an integral) stage process.
\begin{enumerate}
 \item Identify the maximum $c$ of $\phi(t)$.
If it has many maxima, use the largest.
 \item Expand $f(t)$ and $\phi(t)$ about $t=c$, which gives\begin{align*}
        f(t)    & \approx f(c) \quad \text{since it is slowly varying}                   \\
        \phi(t) & = \phi(c) + \frac{(t-c)^2}{2!}\phi^{(2)}(c) + \mathcal{O}\qty((t-c)^3)
       \end{align*}
       where we note that by design $\phi^{\prime}(c) = 0$ and $\phi^{(2)}(c)\leq 0$.
 \item Expand the integration range from $[a,b]$ to $(-\infty, \infty)$, as the error in the integration is exponentially suppressed proportional to the complementary error function.
\end{enumerate}

In practice, to first order this gives the integral
\begin{align*}
 I(x) & = f(c) e^{x\phi(c)}\int_\mathbb{R} e^{(t-c)^2 \frac{x}{2}\phi^{(2)}(c)}\dd{t}          \\
      & = f(c) e^{x\phi(c)}\int_\mathbb{R} e^{-u^2 \frac{x}{2}\abs{\phi^{(2)}(c)}}\dd{u}       \\
      & =f(c) e^{x\phi(c)} \sqrt\frac{2\pi}{\abs{\phi^{(2)}(c)}}\frac{1}{x^{\nicefrac{1}{2}}}.
\end{align*}
This is the first order of an asymptotic expansion
$$I(x) \sim e^{x\phi(c)} \sqrt\frac{2\pi}{\abs{\phi^{(2)}(c)}}\frac{1}{x^{\nicefrac{1}{2}}} \qty[f(c) + \frac{A}{x} + \frac{B}{x^2} + \cdots].$$
\section{Stationary Phase Approximation}
The stationary phase approximation is very similar to Laplace's method, just for integrals of the form $$I(x) = \int_a^b f(t) e^{ix\psi(t)}\dd{t}$$ with all of the same assumptions, and large positive $x$.

The key point here is that the $e^{ix\psi(t)}$ will be strongly oscillatory for large $x$, except around where there is a stationary point in $\psi(t)$.
This allows us to expand around each of the extrema, sum the contributions and get a reasonably accurate approximation to the integral, as away from the extrema the contributions will cancel.

The general first order solution to the integral is
\begin{align*}
 I(x)\sim f(c)e^{ix\psi(c)\pm i\nicefrac{\pi}{4}}\sqrt\frac{2\pi}{x\abs{\psi^{(2)}(c)}}
\end{align*}

\section{The Saddle-Point Approximation}
The next obvious question to ask is ``How do I do all of that, but with a general complex integral?''.
This leads to the Saddle-Point Approximation.

For an integral of the form
\begin{align*}
 I(N) & = \int_C g(z)e^{Nf(z)}\dd{z} \text{ for }N\gg 0
\end{align*}
where $f(z)$ is a generic analytic (in the domain) complex valued-function.
We combine the two previous methods: we look for points where $\Re(f(z))$ is maximised and $\Im(f(z))$ is stationary.
Thus, we require $f^\prime(z)=0$.

\begin{note}
 Stationary points of holomorphic functions are always saddle-points.
This is a consequence of the definition of holomorphicity (with the Cauchy-Riemann equation).
\end{note}

Once we've found the saddle points, we continue in the same vein as the Laplace Method, which the approximations
\begin{align*}
 g(z) & \approx g(z_i) & f(z) & \approx f(z_i) + \frac{1}{2}f^{\prime\prime}(z_i)(z-z_i)^2
\end{align*}
which makes the integrals
$$ I(x) \approx \sum_{z_i} g(z_i)e^{Nf(z_i)} \int_C \exp(\frac{N}{2}f^{\prime\prime}(z_i)(z-z_i)^2)\dd{z}.$$
Now setting $z-z_i=re^{i\phi_i}$ and $f^{\prime\prime}(z_i)=\abs{f^{\prime\prime}(z_i)}e^{i\theta}$ gives us a convergence condition, since
$$I(x)\approx \sum_{z_i} g(z_i)e^{Nf(z_i)} \int_{-\infty}^\infty \exp(\frac{N}{2}\abs{f^{\prime\prime}(z_i)}e^{i\theta}r^2e^{2i\phi_i})e^{i\phi_i}\dd{r}$$ givens us that we want
\begin{align*}
 e^{2 i\phi_i}e^{i\theta}        & = -1                     \\
 \implies \quad 2\phi_i + \theta & = \pi                    \\
 \implies \quad \phi_i           & = \frac{\pi - \theta}{2}
\end{align*} in order to use our nice gaussian integrals.

We need to make a careful choice of contour, such that it goes through each of the saddle points along the `path of steepest descent' in a `positive sense'.
If this sounds vague, it's not being falsely advertised.

\chapter{Stuff that Dirac didn't want named after him}
An applied impulse can be modelled as a sudden spike of applied force ($mv = \int_{t-\epsilon}^{t+\epsilon}F(\tau)\dd{\tau}$), which in the $\epsilon\rightarrow 0$ limit gives us a \emph{Dirac delta distribution} such that $$ F(t) = mv\delta(t-t_0).$$

There are many definitions of this object, all leading from approximations of the Definition, $$ \int_{-\infty}^{\infty} \delta(x) \dd{x} = 1 .$$
We also have the top hat function
\begin{equation*}
 \delta_n^{(1)} (x) \equiv \begin{cases}
  0           & x<-\frac{1}{n}                       \\
  \frac{n}{2} & -\frac{1}{n} \leq x \leq \frac{1}{n} \\
  0           & \frac{1}{n} < x
 \end{cases}
\end{equation*}
which satisfies the Definition,  but isn't terribly useful as it isn't analytic.
However, it does take infinite integrals and make them finite (but not necessarily obviously bounded) which is nice.

Alternatively, defining in terms of a Gaussian gives
$$\delta(x) \equiv \lim_{n\rightarrow\infty}\int_\mathbb{R} \frac{n}{\sqrt{\pi}}e^{-n^2x^2}\dd{x}.$$

There are more, but the point is that they don't really exist, since the limits don't really exist.

\section[Heaviside Thetas out of the Dirac delta]{Heaviside $\Theta$s out of the Dirac-$\delta$}
The Heaviside $\Theta$ function is defined as
\begin{equation*}
 \Theta(t) \equiv \begin{cases}
  1           & t>0                          \\
  \frac{1}{2} & t = 0 \text{ conventionally} \\
  0           & t<0
 \end{cases}
\end{equation*}
which for non-zero $t$ clearly gives us
$$\int_{-\infty}^t \delta(u)\dd{u} = \Theta(t)$$ which (by the fundamental theorem of calculus) immediately gives us $$ \delta(u) = \Theta^\prime (t).$$ You can also do this with integration by parts on $\int_{-\infty}^{\infty}f(t)\Theta^\prime(t)\dd{t}$.

We have an integral representation of the $\Theta$ function,
\begin{alignat*}{2}
 \Theta(x) & = \frac{1}{2\pi i} \int_{-\infty - i\gamma}^{\infty - i \gamma}\frac{e^{ikx}}{k}\dd{k} & \quad\gamma & > 0\intertext{which we can use the relationship derived previously to gain}
 \delta(x) & = \frac{1}{2\pi} \int_{-\infty - i\gamma}^{\infty - i \gamma} e^{ikx} \dd{k}
\end{alignat*}
where the contours are chosen to avoid the poles in the $\Theta$ representation, but there are no poles in the $\delta$ so we can set $\gamma=0$.
The $\delta(x)$ representation is still divergent, so needs another integral in $x$ to make sense.

\part{Ordinary Differential Equations and Where to Find Them}
\chapter{Linear Ordinary Differential Equations}
An $n$\textsuperscript{th} order ODE is a relation
$$ y^{(n)}(x) = F(x, y(x), y^{(1)}(x), \ldots, y^{(n-1)}(x))$$
where $y^{(i)}(x)=\dv[i]{y}{x}$.
If it is linear in $y$ and its derivatives, then the ODE is said to be linear.
An $n$\textsuperscript{th} order linear ordinary differential equation will have $n$ linearly independent solutions and $n$ independent constants of integration (which can be zero).
$$ y(x) = \sum_{j=1}^{n} C_j y_j(x).$$

A general $n$\textsuperscript{th} order ODE can be written as a system of first order equations if we set
\begin{align*}
 y_k             & = \dv[k]{y}{x}      & k & = 0 \,\ldots\, n-1 \\
 \dv{y_k}{x}     & = y_{k+1}(x)        & k & = 0 \,\ldots\, n-2 \\
 \dv{y_{n-1}}{x} & = F(x, \{y_k(x)\}).
\end{align*}
We can also write them formally as
\begin{align*}
 \mathcal{L}y(x) & = f(x)
\end{align*}
where
\begin{align*}
 \mathcal{L} & = \dv[n]{x} + \sum_{k=0}^{n-1} p_{k} \dv[k]{x}.
\end{align*}
If $f(x)=0$ then the ODE is homogeneous.

\section{Linear (in)dependence}
Linear independence is defined as when there is no non-trivial ($\alpha_i \neq 0 \,\forall \alpha_i$) solution for $$\alpha_jy_j(x) = 0$$ (using the Einstein Summation Convention).
We can also differentiate this $n-1$ times to provide constraints for the equations in all of the variables of $F$, to get
$$\alpha_j\dv[i]{y_j}{x} = 0.$$

This is $n$ linear equations in $n$ variables, so we build a Wronskian
\begin{align*}
 W(x) & = W[y_1, y_2, \ldots, y_n] \\
      & \equiv \det
 \begin{vmatrix}
  y_1         & y_2         & \cdots & y_n         \\
  y_1^{(1)}   & y^{(1)}_2   & \cdots & y^{(1)}_n   \\
  \vdots      & \vdots      & \ddots & \vdots      \\
  y_1^{(n-1)} & y^{(n-1)}_2 & \cdots & y^{(n-1)}_n
 \end{vmatrix}.
\end{align*} If the Wronskian is non-vanishing for any $x$ in the domain, this system can be inverted (with respect to $W_{ij}\alpha_i = 0$) to get the independence condition.

The Wronskian of $W(x)$ of any $n$ solutions of the LODE satisfies
$$W^\prime (x) = -p_{n-1}(x)W(x)$$
which has a solution up to a multiplicative constant of
$$W(x) = \exp[-\int^x p_{n-1}(t) \dd{t}].$$ This is Abel's formula, and comes in handy.

\section{Second-Order Linear Homogeneous Ordinary Differential Equations}
With second order ODEs, we can do clever things with Wronsky's determinant.
We notice that if we do
$$\dv{x}(\frac{y_2}{y_1}) = \frac{y_1 y_2^\prime - y_1^\prime y_2}{y_1^2} = \frac{W}{y_1^2}$$ which is very useful \emph{indeed}.
Bringing back Abel's formula, we get an integral equation for the second linearly independent solution if we have one which is
\begin{align*}
 y_2 & = y_1 \int_{x_0}^x \frac{W(t)}{y_1^2(t)} \dd{t}                                            \\
     & =  y_1 \int_{x_0}^x \frac{1}{y_1^2(t)}\exp[-\int^t p_{n-1}(t^\prime) \dd{t^\prime}] \dd{t}
\end{align*}
where $x_0$ is a constant, up to a multiplicative constant, which is dealt with through the boundary conditions.

\chapter{Messers Sturm and Liouville's Form and Mr Green's Functions}
For a second-order linear ODE, we can always write it in the form
$$\mathcal{L}(x)y(x) = f(x)$$
where
$$\mathcal{L}(x) = \dv{x}\qty[p(x)\dv{x}] + q(x).$$
We identify $p(x)$ and $q(x)$ by manipulating
$$y^{\prime\prime} + p_1 y^\prime + p_0 y = g(x)$$
into
$$\qty[\dv{x}(\frac{1}{W(x)}\dv{x}) + \frac{p_0(x)}{W(x)}]y(x) = \frac{g(x)}{W(x)}.$$
This is the Sturm-Liouville form.

\section{Boundary Conditions and Colourful Functions}
Generally, differential equations are defined on a range $a \leq x \leq b$ and boundary conditions given in the form
\begin{align*}
 \alpha y(a) + \beta y(a) & = 0 & \gamma y(b) + \delta y(b) & = 0.
\end{align*}
These are homogeneous ($=0$) and unmixed (each boundary is defined in only equation).

We assume a particular solution of the form
$$ y_p(x) = \int_a^b G(x,x^\prime)f(x^\prime)\dd{x^\prime}$$ where $G(x,x^\prime)$ is a Green's Function.
This leads us to our definition,
$$\mathcal{L}\,G(x,x^\prime)=\delta(x-x^\prime).$$
\subsection*{Finding A Green's Function}
Determining $G(x,x^\prime)$ is a bit of a task.
First, we integrate over an infinitessimal region either side of $x^\prime$ to get
\begin{align*}
 \int_{x^\prime-\epsilon}^{x^\prime + \epsilon}\,\dv{x}[p(x)\dv{G}{x}]+q(x)G(x,x^\prime)\,\dd{x} & = 1.
\end{align*}
By choosing $G(x,x^\prime)$ to be continuous, we get a discontinuity on the first derivatives
\begin{equation*}
 p(x^\prime)\qty[\dv{G(x+0,x^\prime)}{x} - \dv{G(x-0,x^\prime)}{x}] = 1.
\end{equation*}

We then consider the function away from (the already arbitrarily chosen) $x^\prime$, which needs to satisfy the homogeneous equation and the boundary conditions for the region considered.
We choose
\begin{align*}
 G(x,x^\prime) & =
 \begin{cases}
  C_1(x^\prime)y_1(x) & \text{for } x< x^\prime \\
  C_2(x^\prime)y_2(x) & \text{for } x> x^\prime
 \end{cases}
\end{align*}
where the homogenous solutions are chosen such that they solve the boundary condition in that region.

We then fix the constant functions $C_i(x^\prime)$ using these constraints, as
\begin{align*}
 C_1(x^\prime)y_1(x^\prime) - C_2(x^\prime)y_2(x^\prime)               & = 0                      \\
 C_1(x^\prime)y^\prime_1(x^\prime) - C_2(x^\prime)y^\prime_2(x^\prime) & = \frac{1}{p(x^\prime)}.
\end{align*}

The last step is to construct the Green's function using the Heaviside function.
At this point, it's worth drawing a diagram as it's a bit confusing.
The $\Theta(x-x^\prime)$ term is non-zero if $x^\prime<x$ and vice versa, so
\begin{equation*}
 G(x,x^\prime) = C_1(x^\prime)y_1(x)\Theta(x^\prime - x) + C_2(x^\prime)y_2(x)\Theta(x - x^\prime).
\end{equation*}

This process changes subtly if it is an initial value problem.
It being an initial value problem implies that $G(0,t)=\dv{t}\eval{G(t, t^\prime)}_{t=0}=0$, as for $t<t^\prime$ (when the impulse is applied) there is no effect (in a causal system).
The Green's function will be different for a causal system to an anti-causal one.

\section{Normal Forms}
We can separate into other interesting forms by setting $$y(x)=v(x)u(x)$$ which we then chose $v(x)$ to be a known solution.
We can then also choose
\begin{align*}
 v(x) & = \exp[-\frac{1}{2}\int^x P_1(x^\prime) \dd{x^\prime}] \\
      & = W^{\nicefrac{1}{2}}(x)
\end{align*}
which is chosen to eliminate the $u^\prime$ term.
The equation reduces in this form to
$$u^{\prime\prime}(x) + Q(x)u(x) = 0$$ where $$Q(x) = p_0(x) - \frac{1}{2}p_1^\prime(x) - \frac{1}{4}p_1^2(x)$$
which is the Normal or ``Schr\"odinger'' form.

\chapter{One Thousand Coefficients and Variables (Power Series)}
\section{Singularities}
Like complex functions described earlier, differential equations have singularities of the same three types.
We classify the point $x_0$ of an equation
$$y^{\prime\prime} + p(x)y^\prime + q(x)y = 0$$
as
\begin{description}
 \item [ordinary point] if $p(x_0)$ and $q(x_0)$ are both analytic;
 \item [regular singularity] if $p(x_0)$ and/or $q(x_0)$ are not analytic, but $(x-x_0)p(x)$ and $(x-x_0)^2q(x)$ are analytic at $x=x_0$;
 \item [irregular singularity] if $p(x_0)$ and/or $q(x_0)$ are not analytic, and $(x-x_0)p(x)$ and $(x-x_0)^2q(x)$ are not analytic at $x=x_0$.
\end{description}
We test the ``point at infinity'' with $z=\frac{1}{x}$ and let $z\rightarrow 0$.

\section{Taylor-Maclaurin Series Expansions}
For an ordinary point, we use the uniqueness of the power series to say
$$ y(x) = \sum_{m=0}^\infty c_m(x-x_0)^m $$
which gives a convergent series with radius of convergence up to the nearest singular point (by analytic continuation).
Substitution and index adjustment will give a recursion relation, with which is then used to compare coefficients.

\section{Frobenius Expansion about a Regular Singular Point}
A regular singular point cannot be Taylor expanded about.
We can try a Frobenius series though, which is in the form
\begin{equation*}
 y(x)= (x-x_0)^\alpha \sum_{m=0}^\infty c_m(x-x_0)^m \text{ for }c_0 \neq 0
\end{equation*}
The parameter is derived by comparing coefficents of the lowest powers of $x$ which gives an \emph{indicial equation} (or many), where we also use the definition that $c_0\neq 0$.
This will often give you a quadratic equation.
The nature of $\alpha_+ - \alpha_-$ dictates how many series solutions are found with this method.
If it is an integer, there are two independent solutions.
Otherwise, the second solution will be of the form
$$ y_2(x) = \ln(x-x_0)y_1(x) + C(x)(x-x_0)^{\alpha}.$$

\section{Expansion about irregular points}
We begin by putting the differential equation into normal form ($y(x)=v(x)u(x)$ with $v(x)=\exp[-\frac{1}{2}\int^x P_1(x^\prime) \dd{x^\prime}]$).
The normal form gives us lots of information, and can reasonably easily then be approximated (for example $\abs{x} \gg 1$).
To expand about infinity, we do the usual trick of $z = \frac{1}{x}$ as $z\rightarrow 0$.

\chapter{Unfogging the (special) Function}
\section{Legendre Polynomials}
Legendre's equation is
\begin{equation*}
 (1-x^2)y^{\prime\prime}(x) - 2xy^{\prime}(x) + n(n+1)y(x)=0
\end{equation*}
which we can use a Taylor-Maclaurin series to solve which gives us a recurrence relation
\begin{equation*}
 c_{m+2} = \frac{(m-n)(m+n+1)}{(m+2)(m+1)}c_m
\end{equation*}
and we note that for that for any integer $n$, either the even or odd series will be finite, which we call the Legendre polynomial.
Solving the recurrence relation (through not inconsiderably faffery) shows that these solutions are
\begin{align*}
 P_n(x) & = \sum_{r=0}^{\left[\nicefrac{n}{2}\right]}\frac{(-1)^r(2n-2r)!}{2^n \,r!\, (n-r)!\, (n-2r)!}x^{n-2r} \\
        & = \sum_{r=0}^{\left[\nicefrac{n}{2}\right]} \frac{(-1)^r}{2^n\,r!\,(n-r)!}\dv[n]{x}x^{2n-2r}          \\
        & = \frac{1}{2^n\,n!}\dv[n]{x}\sum_{r=0}^{n}\frac{n!}{r!(n-r)!}(-1)^r(x^2)^{n-r}                         \\
        & = \frac{1}{2^n\,n!}\dv[n]{x} (x^2 -1)^n.
\end{align*}
which is Rodrigues' formula.

The integral formula gives us a further extension to this where
\begin{align*}
 P_n(x) & = \frac{1}{2^n}\frac{1}{2\pi i}\oint\frac{(t^2-1)^n}{(t-x)^{n+1}}\dd{t}
\end{align*}
which is \emph{Schl\"afli's integral representation}.

\section{Generating functions}
The generating function is useful for solving relations.
We let
$$F(h, x) = \sum^\infty_{n=0} h^n P_n(x)$$ where if we already know a closed integral for $P_n$.
For the Legendre Polynomials, a rearrangement, the geometric series and a contour integral lead to the solution that
$$F(h, x) = \frac{1}{\sqrt{1-2xh+h^2}}.$$

From this expression, we can derive a recurrence relation (which is the main objective of the exercise).
We differentiate with respect to $h$ to get
\begin{align*}
 \pdv{F}{h}            & = \sum_{n=0}^{\infty} nh^{n-1}P_n \\
                       & = \frac{x-h}{1-2hx+h^2}F          \\
 (1-2hx+h^2)\pdv{F}{h} & = (x-h)F
\end{align*}
with which we compare coefficients of $h$ (since that's the arbitrary variable).
This gives us a generating function form of
$$ P_n(x) = \frac{1}{n!} \eval{\pdv[n]{f(h,x)}{h}}_{h=0}. $$

Usefully, we can work from here towards showing orthogonality.
Legendre polynomials are orthogonal with relation
$$\int_{-1}^{1}P_m(x)P_n(x)\dd{x}=\frac{2}{2n+1}\delta_{mn}.$$

\section{Bessel Functions}
Bessel's equation is
$$x^2y^{\prime\prime} + xy^\prime +(x^2-n^2)y$$ which has a regular singular point at $x=0$ and thus is solved by the method of Frobenius.
This gives us the Bessel functions (of the first kind)
\begin{equation*}
 J_n(x)= \sum_{p=0}^\infty \frac{(-1)^p}{p!\,(p+n)!}\qty(\frac{x}{2})^{n+2p}.
\end{equation*}
which is invariant under $n\rightarrow -n$ so we take
\begin{equation*}
 J_{-n}(x) = (-1)^nJ_n(x).
\end{equation*}

We expand the generating function in a Laurent series on $h$ as
\begin{equation*}
 F(x,h) = \sum_{n\in\mathbb{Z}} h^n J_n(x).
\end{equation*}
With some manipulation of factorials\footnote{which I'm currently unsure of}
you can get two recurrence relations,
\begin{align*}
 J_{n-1} + J_{n+1} & = \frac{2n}{x}J_n \\
 J_{n-1} - J_{n+1} & = 2 J_n^\prime
\end{align*}
which we can multiply through by $h^n$ and sum $\forall n \in \mathbb{Z}$ to get
\begin{align*}
 \qty(h + \frac{1}{h})F(x,h) & = \frac{2h}{x} \pdv{F(x,h)}{h}               \\
 \implies\qquad F(x,h)       & =\phi(x)\exp(\frac{x}{2}\qty[h-\frac{1}{h}])
\end{align*}
Using the other recursion relation fixes $\phi(x)$ as a constant, which we usually choose as $\phi=1$.

We can invert the generating function to get an expression for $J_n(x)$, which is typically tricky.
We can use the residue theorem to produce
\begin{equation*}
 J_n(x)=\frac{1}{2\pi i}\oint\frac{F(t,x)}{t^{n+1}}\dd{t}
\end{equation*}
which evaluation gives us a complicated but approximately soluble integral, with the contour circling the origin.

The Bessel function generalises with the help of $\Gamma$ functions, but the integral representation is the same, with a Hankel contour substituting for the closed contour.
\part{A Beginner's Guide to Transformations}
\chapter{Fourier Transforms}
\section{Fourier Series}
The complex Fourier series is defined as
\begin{equation*}
 f(x) = \sum_{n\in\mathbb{Z}} C_n\exp(\frac{in\pi x}{L})
\end{equation*}
with
$$C_n = \frac{1}{2L}\int_{-L}^L f(x)\exp(-\frac{in\pi x}{L})$$ which is derived from the integral definition of the Kronecker $\delta_{ij}$.
You can also define $\sin$ and $\cos$ series which are just the real/imaginary parts of the complex series, with limits chosen carefully to avoid discontinuity.

\section{Fourier Transforms}
We define
$$\fourier{f(x)} = \int_{-\infty}^{\infty}f(x)e^{-ikx}\dd{x}$$ which is often denoted $\tilde{f}(k)$.
The Fourier transform is well defined and uniformly convergent for all square-integrable functions (where $f(x)\rightarrow 0$ as $x\rightarrow\infty$).
The inverse transform is
\begin{equation*}
 \inversefourier{g(k)} = \frac{1}{2\pi}\int_{-\infty}^{\infty}g(k)e^{ikx}\dd{k}.
\end{equation*}
There are many conventions on the usage of $e^{\pm ikx}$ and where the $\nicefrac{1}{2\pi}$ factor goes.
This is entirely arbitrary, so long as they nicely pair off to $\fourier{\inversefourier{f(x)}}=f(x)$

This generalises to $n$-dimensional spaces by
\begin{align*}
 \fourier{f(\vec{x})} & = \int_{\mathbb{R}^n} f(\vec{x})e^{i\vec{k}\vdot\vec{x}}\dd[n]{x} & \inversefourier{\tilde f(\vec{k})} & = \frac{1}{(2\pi)^n}\int_{\mathbb{R}^n} \tilde f(\vec{k}) e^{i\vec{k}\vdot\vec{x}} \dd[n]{k}
\end{align*}
\subsection{General Properties}
We can derive by integration by parts that
\begin{equation*}
 \dv[n]{x}\fourier{f(x)} = (ik)^n\fourier{f(x)}
\end{equation*} which is really useful.

A Gaussian Fourier transforms to another Gaussian of inverse width in 3-D.
This gives us (among other things) the Heisenberg Uncertainty Principle.

\section{Convolutions}
The convolution of two functions $f_1(x)$ and $f_2(x)$ is defined as
\begin{align*}
 (f_1 \ast  f_2)(x) = \int_{-\infty}^\infty f_1(y)f_2(x-y)\dd{y}.
\end{align*}

We can also show the convolution theorem,
\begin{align*}
 \fourier{f_1 \ast  f_2} & = \fourier{f_1}\fourier{f_2}        \\
 \fourier{f_1f_2}        & = \fourier{f_1} \ast \fourier{f_2}.
\end{align*}

\section{Solving ODEs with Fourier Transformations}
Due to the differentiation property, we can often solve ODEs much easier with a Fourier transform.
Fundamentally, for an ODE of the form $\mathcal{L}y=f(x)$, we do
\begin{align*}
 \mathcal{L}\tilde y (\omega) & =\tilde f(\omega)\intertext{which gives us a algebraic expression}
 A(\omega)\tilde{y}(\omega)   & =\tilde{f}(\omega)\intertext{which you can then recognise as }
 \tilde{G}(\omega)            & = \frac{1}{A(\omega)}\intertext{giving the convolution}
 y(t)                         & = \int_{-\infty}^\infty G(t-t^\prime)f(t)\dd{t}.
\end{align*}
Clearly, the difficult bit here is inverting the Green's function.
We also have to assume that $\lim_{t\rightarrow\infty}G(t)=0$.
We can combine Jordan's Lemma with contour integration and the residue theorem.

If we have inhomogeneous boundary conditions, we have to adjust our expression for $\tilde{y}(\omega)$ so that we aren't at any point dividing by zero.
This is done by adding delta functions at the boundaries
such that
$$\tilde{y}(\omega) = \frac{1}{A}\tilde{f}(\omega) + 2\pi A\delta(\omega-\omega_+)+2\pi B\delta(\omega-\omega_-).$$

\chapter{Laplace Transformations}
The Laplace transform of a function is defined as
$$\laplace{f(t)}=F(s)=\int_{-\infty}^\infty f(t)e^{-st}\dd{t}$$
where $s$ is ``sufficiently large'' and $t>0$ for the sake of convergence.

The inverse transform is not quite as clear cut as in the Fourier case.
It is given by the Bromwich Inversion Formula
\begin{equation*}
 f(t) = \frac{1}{2\pi i}\int_{\gamma-i\infty}^{\gamma+i\infty}F(s)e^{st}\dd{s}
\end{equation*} where $\gamma$ is chosen to be to the right of all of the ``interesting'' points in the function.
This allows us to find the inverse using the residue theorem.
\section{Properties}
Similarly to the Fourier transform, the Laplace transform has convenient differentiation properties, with \begin{align*}
 \laplace{f^{\prime}(t)}       & = sF(s) - f(0)                  \\
 \laplace{f^{\prime\prime}(t)} & = s^2F(s) - sf(0) - f^\prime(0) \\
 \dv{s}\laplace{f(t)}          & = -\laplace{tf(t)}.
\end{align*}

This allows us to solve initial value problems.
We simply need values for $f(0)$ and $f^\prime(0)$.

It also follows the convolution theorem the same as Fourier Transforms, $$\laplace{f*g}=\laplace{f}\laplace{g}.$$

In the large $t$ limit, the behaviour of a function to be inverse transformed is given by the pole closest to the $\gamma$ contour.
If there's a branch point, a Hankel-like contour is used.
If a branch point is the closest to the contour, we expand as an asymptotic series, then invert term by term.

\chapter{Other Transforms}
Many other transforms are possible, based on some kernel.
\begin{center}
 \begin{tabular}{lll}\toprule
  Transform         & Forward                                          & Inverse                                                                       \\\midrule
  Fourier Transform & $g(k)=\int_{-\infty}^{\infty}f(x)e^{-ikx}\dd{x}$ & $f(x)=\frac{1}{2\pi}\int_{-\infty}^{\infty}g(k)e^{ikx}\dd{k}$                 \\
  Laplace Transform & $F(s)=\int_0^\infty f(t)e^{-st}\dd{t}$           & $f(t)=\frac{1}{2\pi i}\int_{\gamma-i\infty}^{\gamma+i\infty}F(s)e^{st}\dd{s}$ \\
  Mellin Transform  & $\phi(z)=\int_0^\infty f(t)t^{z-1}\dd{t}$        & $f(t) = \frac{1}{2\pi i}\int_{c-i\infty}^{c+i\infty}\phi(z)t^{-z}\dd{z}$      \\\bottomrule
 \end{tabular}
\end{center}

\section{Generalised Integral Transform Method --- Laplace's (other) Method}
We develop a generalised integral transform method to get
\begin{enumerate}
 \item a systematic way to solve linear ODEs
 \item analytic continuation automatically incorporated.
\end{enumerate}
It forms a bridge between differential equations and operators in the Hilbert space.

We form a linear ODE of polynomial coefficients in the form
\begin{equation*}
 \mathcal{L}_z u(z) = 0
\end{equation*} and seek solutions of the form
\begin{align*}
 u(z) & = \int_C K(z,s) v(s) \dd{s}
\end{align*} for some kernel $K(z,s)$ and contour $C$.

Examples of useful kernels include
\begin{description}
 \item [Fourier] $e^{izs}$,
 \item [Laplace] $e^{zs}$,
 \item [Euler] $(s-z)^\mu$,
 \item [Mellin] $z^{-s}$,
 \item [Bessel] $sJ_n(zs)$ \ldots
\end{description}

Now we can start on the general method.
We choose $K(z,s)$ such that there exists a particularly simple differential operator $M_s$ with
\begin{equation*}
 \mathcal{L}_z K(z,s) = M_s K(z,s).
\end{equation*}
We can show this schematically.
We start with
\begin{align*}
 \mathcal{L}_z u(z) & = \mathcal{L}_z\int_C K(z,s)v(s)\dd{s}                                                 \\
                    & = \int_C \mathcal{L}_z K (z,s)v(s)\dd{s}                                               \\
                    & = \int_C \qty[M_s K(z,s)]v(s) \dd{s}\intertext{which by integration by parts gives us}
                    & = \qty[R(z,s)]_C - \int_C K(z,s)M_s^\dagger v(s) \dd{s}                                \\
                    & = 0
\end{align*}
by the original differential operator and that $M_s\{K(z,s)v(s)\} = [M_s K]v + K(M_s^\dagger v)$.

The `game' here is to get rid of the $z$-dependence by having $M_s$ (or $M_s^\dagger$) acting on $v(s)$ alone.
With this in mind, we choose $v(s)$ such that it soves $M_s^\dagger v(s)= 0$ and choose the contour $C$ such that $\eval{R(z,s)}_C$ vanishes.
This makes a lot more sense in practice!

\part{The Dark Forces: A Guide to Partial Differential Equations}
\chapter{Classification of PDEs}
A first order PDE is in the form
\begin{equation*}
 F(x,y,u,u_x,u_y)=0
\end{equation*}
where $u_x=\pdv{u}{x}$.
This generalises to second order with
\begin{equation*}
 F(x,y,u,u_x,u_y,u_{xx},u_{xy},u_{yy})=0.
\end{equation*}
A linear PDE is linear in $u$ and all its derivatives, but not necessarily in $x$ or $y$.

In general, the solution of a PDE is an arbitrary function, and the ``constants of integration'' from ODE theory become ``functions of integration'' in all but the integrated over variables.
We find these through imposition of boundary conditions.

Considering the first order PDE
\begin{equation*}
 au_x + b u_y = 0
\end{equation*}
for $a$, $b$ constant, we can interpret this as a directional derivative
\begin{equation*}
 \qty(\pmqty{a\\b} \vdot \grad)u = 0
\end{equation*} which means that along lines parallel to the direction the function is constant --- these are referred to as characteristics.
For this first order PDE the characteristics are along the lines defined by
$$bx-ay=m$$ which makes the the final general solution of the form $u=f(bx-ay)$.
\chapter{Second Order Partial Differential Equations}
\begin{table}[htbp]
 \centering
 \begin{tabular}{l c c}
  \toprule
  Equation      & 4D Version                                                             & 2D Version                                                   \\
  \midrule
  Wave          & $\laplacian u - \frac{1}{c^2}\pdv[2]{u}{x}=0$                          & $u_{xx} - \frac{1}{c^2}u_{tt}=0$                             \\
  Laplace       & $\laplacian u = 0$                                                     & $u_{xx} + u_{yy} = 0$                                        \\
  Diffusion     & $\laplacian u - \frac{1}{\kappa} u_t = 0$                              & $u_{xx} - \frac{1}{\kappa} u_t = 0 $                         \\
  Schr\"odinger & $\qty(-i\hslash \pdv{t} + \frac{\hslash^2}{2m}\laplacian + V)\psi = 0$ & $\qty(ih\pdv{t} + \frac{\hslash^2}{2m}\pdv[2]{x} + V)\psi=0$ \\\bottomrule
 \end{tabular}
 \caption{Some homogeneous partial differential equations}\label{tbl:homogenouspde}
\end{table}

Shown in table \ref{tbl:homogenouspde} are some important examples of homogeneous PDEs in physics.
We can also define these as inhomogeneous versions.

A general class of second order PDE in the form $$A(x,y)u_{xx} + 2B(x,y)u_{xy} + C(x,y)u_{yy} = D(x,y,u_x,u_y)$$
requires more than a point to define the boundary conditions, so we define a boundary \emph{function} over some curve $\Gamma$.
We can form a linear system by considering the differentials $\frac{\partial^2 u}{\partial x \partial s}$ and $\frac{\partial^2 u}{\partial y \partial s}$, which we relate with the parameterisation of $\Gamma$.

On the characteristics, the linear system is not invertable.
This is how they are defined.



\section{Normal Forms}
We can change variables with impunity in our PDEs.
A good choice is to change to `normal' co-ordinates where the cross term $\frac{\partial^2 u}{\partial x \partial y}$ vanishes.

From this process, we get the forms shown in table \ref{tbl:normaltransformations}, where $$\alpha_\pm(x,y) = y - \frac{B\pm\sqrt{B^2 - AC}}{A}x$$ which should be constant.
The number and type of each characteristic will show display which type of form should be used.

\begin{table}[htb]
 \centering
 \begin{tabular}{lllll}
  \toprule
  Class      & \multicolumn{2}{c}{Change of Variables} & New Equation                        & Type                                                           \\
  \midrule
  Elliptic   & $v=\frac{1}{2}(\alpha_+ + \alpha_-)$    & $w=-\frac{i}{2}(\alpha_+-\alpha_-)$ & $\pdv[2]{u}{v} + \pdv[2]{u}{w}=\tilde{D}$ & ``Laplace-like''   \\
  Hyperbolic & $v=\frac{1}{2}(\alpha_+ + \alpha_-)$    & $w=\frac{1}{2}(\alpha_+-\alpha_-)$  & $\pdv[2]{u}{v} - \pdv[2]{u}{w}=\tilde{D}$ & ``Wavelike''       \\
  Parabolic  & $v=\alpha$                              & $w=x$                               & $\pdv[2]{u}{w}=\tilde{D}$                 & ``Diffusion-like'' \\\bottomrule
 \end{tabular}
 \caption{Normal Transformations of Differential Equations}\label{tbl:normaltransformations}
\end{table}

\section{Boundary Conditions}
There are many types of boundary conditions
\begin{description}
 \item [Dirichlet] Defining on a curve the value of the function ($u$)
 \item [Neumann] Defining on a curve the first derivative of the function ($\pdv{u}{n}$)
 \item [Cauchy] Defining $u$ and $\pdv{u}{n}$ on $\Gamma$.
\end{description}
For hyperbolic equations, we use Cauchy conditions, for elliptics, we have tu use Dirichlet or Neumann on a closed curve, and Parabolics require that you define one or other of Dirichlet or Neumann, but not both.


\chapter{The Wave Equation}
The simplest hyperbolic equation is the wave equations$$u_{xx}-\frac{1}{c^2}u_tt=0$$ for which the characteristics are given by
$$\alpha_\pm = x \pm ct$$ (which is where the concept of `preserving the interval' comes from).
The general solution for the wave equation is
$$u(x,t) = f(x-ct) + g(x+ct)$$ which shows the propagation along the characteristic.

We can also solve it using the Fourier Transform, for
$$F(k,t) = \int_{-\infty}^{\infty} u(x,t) e^{-ikx} \dd{x}$$ which gives
\begin{align*}
 -k^2 F(k,t)            & = \frac{1}{c^2}\pdv[2]{F}        \\
 \implies \qquad F(k,t) & = A(k)\cos(kct) + B(k)\sin(kct).
\end{align*}

The transformed basis gives our boundary conditions as
\begin{align*}
 u(x,0)               & = a(x)         & u_t(x,0) & = b(x)         \\
 \implies\quad F(k,0) & = \tilde{a}(k) & F_t(k,0) & = \tilde{b}(k) \\
                      & = A(k)         &          & = kcB(k)
\end{align*}
Thus
$$ F(k,t) = \tilde{a}(k)\cos(kct)+ \frac{\tilde{b}}{kc}\sin(kct). $$

Inverse transforming, we notice a possible application of the convolution theorem.
This gives
$$ u(x,t) = \frac{1}{2}[a(x - ct) + a(x+ct)] + \frac{1}{2c} \int_{x-ct}^{x+ct} b(x^\prime)\dd{x^\prime} $$

\section{Dealing with inhomogeneity}
The define the `D'Alembertian'
$$\square = \laplacian - \frac{1}{c^2}\pdv[2]{t}$$ which means that
\begin{align*}
 \square u(x,t)                                 & = f(\vec{x},t)\intertext{is the inhomogeneous wave equation.
We define the multivariate Green's function with}
 \square G(\vec{x},t; \vec{x}^\prime, t^\prime) & = \delta(\vec{x} - \vec{x}^\prime) \delta(t-t^\prime)
\end{align*}
which we can find via the Fourier Transform method.
In Fourier space
\begin{align*}
 \fourier{G(\vec{x},t)}                    & = \tilde{G}(\vec{k},w)                                                      \\
                                           & = \int \dd[3]{x}\int \dd{t} G(\vec{x},t)e^{-(\vec{k}\vdot\vec{x}-\omega t)} \\
 \implies \qquad \tilde{G}(\vec{k},\omega) & = \frac{c^2}{\omega^2 - k^2c^2}
\end{align*} which we must now invert.

This is not a trivial task, but we face it with confidence and the knowledge which we've gained over the previous sections.
We notice that at some point a contour integral will appear, with two poles at $\omega=\pm kc$.
We must choose which half plane we close the contour in.
This is akin to choosing the causal or anti-causal solution, radiating forwards or backwards in the $t$ axis.
The choices of which poles to include should be clear from the boundary conditions.

A lot of hard work later, we get
\begin{equation}
 u_p(\vec{x},t) = -\frac{1}{4\pi}\int \frac{f(\vec{x}^\prime), t- \frac{1}{c}\abs{\vec{x}-\vec{x}^\prime}}{\abs{\vec{x}-\vec{x}^\prime}}\dd[3]{x^\prime}.
\end{equation}

\chapter{The Diffusion Equation}
The canonical example of a parabolic equation, in one spacial dimension the equation reads
$$\pdv[2]{u}{x} - \frac{1}{D}\pdv{u}{t}=0$$
where $D$ is the diffusion constant.

There are two types of boundary conditions for these equations,
\begin{enumerate}
 \item we specify $u(x,t_0)$ (usually $t_0 = 0$) for all $x\in \mathbb{R}$
 \item we specify $u(x,t_0)$ over some range beginning or ending at finite $x$ (e.g.~$x\in[a,b]$).
Then, in addition, we specify $u(a,t)$ or $u^\prime (a,t)$ on the finite $x$ boundary.
\end{enumerate}

For the infinite domain, we solve this the same way as we have in the past, with Fourier Transform to find a Green's function.

For a finite $x$ boundary, we consider a semi-infinite rod.
We give the boundary conditions as $u(x,0)$ given for $x>0$ and $u(0,t)=f(t)$, otherwise known as knowing the initial state of all of the rod, and how the end of the rod changes (often this is constant).

We can do a $\sin$ or $\cos$ transform as in the previous styles with the Fourier, or use a Laplace transform.
This leads to a nontrivial Laplace inversion, but it is useful to note that $$\laplace{\erfc\qty(\frac{a}{2\sqrt{t}})} = \frac{\exp(-a\sqrt s)}{s}$$
and that the final solution is an asymptotic series
$$T(x,t)\sim T_0 \qty[1-\frac{x}{\sqrt{\pi D t^{\nicefrac{1}{2}}}} + \cdots].$$

\chapter{Laplace's and Poisson's Equations}
For our canonical elliptic operator, we consider Laplace's Equation
\begin{align*}
 \laplacian u(\vec{x}) & = 0 \intertext{and the inhomogeneous equivalent}
 \laplacian u(\vec{x}) & = \rho(\vec{x}).
\end{align*} for both of which we must define closed boundary conditions in either a Dirichlet or Neumann style.

\section{Homogeneous Conditions}
In the homogeneous form, we mean $u(\vec{x})=0$ when $\vec{x}\in S$ where $S$ is a closed curve in $\mathbb{R}^3$.
We construct, as usual, a Green's function, which we assume to have the same boundary conditions as $u$.
This gives us
$$u(\vec{x}) = \int_V G(\vec{x},\vec{x}^\prime)\rho(\vec{x}^\prime)\dd{V^\prime}$$ which can be solved in the usual electrostatics method using the divergence theorem for $G(\vec{x}, \vec{x}^\prime)$.

The solution in this case (spherical integral for $\lim_{\abs{\vec{x}}\rightarrow \infty}G(\vec{x},\vec{x}^\prime) = 0$)
\begin{align*}
 G(\vec{x},\vec{x}^\prime) & = -\frac{1}{4\pi \abs{\vec{x}-\vec{x}^\prime}}
\end{align*}
which is known as the `fundamental' solution of the Laplacian.

\section{Inhomogeneous Boundary Conditions}
Solving this is less trivial (though more general, and includes the homogenous case).
We use Green's Theorem
\begin{align*}
 \int_V\qty(\phi \laplacian \psi - \psi \laplacian \phi)\dd{V} & = \int_S \qty(\phi \grad \psi - \psi \grad \phi)\vdot \hat{n}\dd{S} \\
 \implies \quad \int_V \quad[u(\vec{x})\laplacian G(\vec{x},\vec{x}^\prime) - G(\vec{x},\vec{x}^\prime)\laplacian u(\vec{x})]\dd{V}  \\= \int_S [u(\vec{x})\pdv{n}G(\vec{x},\vec{x}^\prime)
                                                               & - G(\vec{x},\vec{x}^\prime) \pdv{u(\vec{x})}{n}]\dd{V}
\end{align*}
which we rearrange, with an application of the definition of the Green's function and of Shakespeare's Theorem to get
\begin{equation*}
 u(\vec{x}) = \int_V G(\vec{x},\vec{x}^\prime)\rho(\vec{x}^\prime)\dd{V^\prime} + \int_S \qty[u(\vec{x})\pdv{n}G(\vec{x},\vec{x}^\prime) - G(\vec{x},\vec{x}^\prime) \pdv{u(\vec{x})}{n}]\dd{S^\prime}
\end{equation*}
for which the two forms of boundary conditions are catered for with the two terms in the surface integral.

In general, the Green's function for these will be in the form
$$G(\vec{x},\vec{x}^\prime) = F(\vec{x},\vec{x}^\prime) + H(\vec{x},\vec{x}^\prime)$$ where $F(\vec{x},\vec{x}^\prime)$ is the fundamental solution found previously.

\end{document}
